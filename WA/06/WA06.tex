
%%%%%%%%% MASTER -- compiles the summary, description, and reference sections

%%%\documentclass[11pt,onecolumn]{IEEEtran}

\documentclass[10pt]{article}


%\usepackage{course}
\usepackage{pst-all,amssymb,amsmath,amsthm,cite,url,float}
\usepackage[pdfpagelayout=OneColumn]{hyperref}

%\usepackage[usenames, dvipsnames]{color}

%\HideNotes


%\def\handout{Written Assignment 01}  % Define the title of the document. For other parameters, plz see course.sty

\newcommand{\setup}{\mathsf{Setup}}
\newcommand{\gen}{\mathsf{KeyGen}}
\newcommand{\enc}{\mathsf{Enc}}
\newcommand{\dec}{\mathsf{Dec}}
\newcommand{\param}{\mathsf{param}}
\newcommand{\pk}{\mathsf{pk}}
\newcommand{\sk}{\mathsf{sk}}
\newcommand{\adv}{\mathsf{Adv}}
\newcommand{\tab}[1]{\hspace{.2\textwidth}\rlap{#1}}


% Add space between paragraphs, no indents
\setlength{\parskip}         {1ex plus 4pt minus 2pt}
\setlength{\parindent}      {0em}

% Add package for underline command \uline to work
%\usepackage{ulem}

% Set the document margins to conform to NSF standards
\setlength{\topmargin}{-.5in}
\setlength{\textheight}{9.0in}
\setlength{\textwidth}{6.5in}
\setlength{\headheight}{.25in}
\setlength{\headsep}{.25in}
\setlength{\oddsidemargin}{0in}
\setlength{\evensidemargin}{0in}

\begin{document}
\bibliographystyle{IEEEtran}

%The purpose of this document is to codify the logistics needed for students. This will be summarized in the first day of the class.




\begin{center}
	\textbf{CSE 40622 Cryptography, Spring 2018\\Written Assignment 06 (Section 1 in Lecture 15-17)}
\end{center}


Name: \textbf{Jasmine Walker}




\begin{enumerate}
	\item (\textbf{Hard}, 20 pts, page 4) Use Chinese Remainder Theorem to prove RSA encryption works correctly even if $\gcd(m, n) \neq
	1$ where $m$ is the message to be encrypted and $n$ is the RSA modulus $n = pq$.
	
	\textbf{Answer:} \newline Since $\gcd(m, n) \neq 1$, we can't use Euler's Theorem to make the conclusion that $m^{\varphi(n)} \equiv 1 \pmod n$. So, from the decryption of cipher $c$, we have $c^d \equiv m^{ed} \equiv m^{1+k\varphi(n)} \equiv m\cdot(m^{\varphi(n)})^k \pmod n$, and we must prove $m\cdot(m^{\varphi(n)})^k \equiv m \pmod n$. We apply CRT to this term: 
    \begin{center}
    $m\cdot m^{\varphi(n)k} \equiv a_1 \pmod p$
    
    $m\cdot m^{\varphi(n)k} \equiv a_2 \pmod q$
    \end{center}
    Since $\gcd(m, n) \neq 1$, we know that $m$ is either divisible by $p$, $q$, or $n$. If $m$ is divisible by $n$, then $m$ is equal to $0 \mod n$, which will work in RSA decryption because $0$ to the exponent of anything will equal $0$.
    
    For the other cases, let's assume $p$ is the divisor of $m$ \textbf{without loss of generality}. If $m$ is divisible by $p$, then $m \mod p = 0$ and the remainder $a_1$ will equal $0$ no matter the exponent of $m$. Since $p$ and $q$ are prime, we can apply Proposition 1 from Lecture 15-17 to the $\mod q$ term:
    \begin{center}
    $m\cdot m^{\varphi(n)k} \equiv m\cdot m^{\varphi(pq)k} \equiv m\cdot m^{\varphi(p)\varphi(q)k} \equiv a_2 \pmod q$
    \end{center}
    Because $\gcd(m,q) = 1$, we can apply Euler's Theorem to this term and get 
    \begin{center}
    $m\cdot m^{\varphi(p)\varphi(q)k} \equiv m\cdot m^{\varphi(q)(\varphi(p)k)} \equiv m \pmod q$
    \end{center}
    We are left with two terms, 
    \begin{center}
    $m\cdot m^{\varphi(n)k} \equiv x \equiv 0 \equiv a_1 \pmod p$
    
    $m\cdot m^{\varphi(n)k} \equiv x \equiv m \equiv a_2 \pmod q$
    \end{center}
    We can apply CRT's formula of $x$ to get the value of the element in $Z_n$ that will satisfy both these congruences: 
    \begin{center}
    $x = a_1\cdot q\cdot q^{-1}_p + a_2\cdot p\cdot p^{-1}_q \mod n$
    \end{center}
    We know that $a_1 = 0$, so the first term is nulled:
    \begin{center}
    $x = a_2\cdot p\cdot p^{-1}_q \mod n$
    \end{center}
    We also know that $p \cdot p^{-1}_q$ is equivalent to saying $p \cdot p^{-1}_q = 1 + kq$ for some integer $k$. So the formula is now 
    \begin{center}
    $x = a_2\cdot p\cdot p^{-1}_q \mod n = a_2\cdot (1+kq) = a_2+a_2kq \mod n$
    \end{center}
    By inspection, we can see that $x$ has a remainder $a_2$ if mod q is applied, and a remainder $0$ if mod p is applied. CRT states there is only one unique element in $\mathbb{Z}_n$ that can satisfy both requirements, and since we know $m \mod q = a_2$ and $m \mod p = 0$ from above, we already know what it is: $m$. Thus, $m\cdot m^{\varphi(n)k} \equiv x \equiv m \pmod n$, so RSA will work even if $\gcd(m, n) \neq 1$.
	
	
	\bigskip\item (20 pts, page 7) Based on the ideas in Section 1.3.1, research (\textit{i.e.,} by Googling) how Miller-Rabin test works, and describe the algorithm
	with your own language or pseudocode (only one is necessary).
	
	\textbf{Answer:}\newline
     I researched this answer on Wikipedia: for an integer $n$ and an integer $a\in \mathbb{Z}_n - \{0,1,n-1\}$, $a^{n-1}\mod n$ must equal $1$ if $n$ is prime. The square roots of $1$ must be either $1 \mod n$ or $-1 \mod n$ if $n$ is prime. We can use these facts to our advantage in Miller-Rabin primality tests. First, $n-1$ must be factored into the form $2^s\cdot d$, where $d$ is an odd number indivisible by $2$. Then we enter into a for loop of $k$ iterations, $k$ being the specifier for how accurate the primality test is.
    
    \begin{tabbing}
    for \= k loops: \\
    \> x := $a^d \mod n$ \\
    \> if x \= == $1$ or x == $n-1$: \\
    \> \> continue \\
    \> for s - 1 loops: \\
    \> \> x := $x^2 \mod n$ \\
    \> \> if x \= == $1$: \\
    \> \> \> return "n is not prime" \\
    \> \> if x == $n-1$: \\
    \> \> \> continue (back to k loops) \\
    \> return "n is not prime" \\
    return "n is probably prime" \\
    \end{tabbing}
    
	\bigskip\item (\textbf{Hard}, 20 pts, page 7) If $a\in \mathbb{Z}_n$ with an RSA modulus $n=pq$ satisfies $a^{n - 1} \mod n = 1$, $a$ may be useful in factoring $n=pq$.
    Explain why this is so.
    \begin{itemize}
    \item Hint: Reading Section 2.4.4 in Lecture 03-05 will be helpful.
    \end{itemize}
	
	\textbf{Answer:}
    
    We can use the steps of the Miller-Rabin primality test to find the factors of $n = pq$. 
    If we are given an $a$ such that $a^{n - 1} \mod n = 1$, we first factor $n-1$ into the
    form $2^s \cdot d$, $d$ being an odd number. Then we compute $x$ := $a^d \mod n$, and 
    check to see if this number is congruent to $1 \mod n$ or $-1 \mod n$. If it is, it is useless. 
    If it is not, then we square $x$ into $x^2$ and check $x^2$ to see if it congruent to $1 \mod n$ or
    $-1 \mod n$. If it is not equal to either, we square it again into $x^{2\cdot 2} \mod n$ and check 
    it against $-1 \mod n$ or $1 \mod n$ again. If some $x^{2m} \mod n$ is found to be equal to $-1 \mod n$,
    then it is useless. But if there is some $x^{2m} \equiv 1 \pmod n$, then we can factor $n=pq$.
    If $x^{2m} \equiv 1 \pmod n$, and one if its square roots is $x^{2(m-1)}$ that is not equal to $-1 \mod n$
    or $1 \mod n$, then either $x^{2(m-1)} \equiv 1 \pmod p$, $x^{2(m-1)} \equiv -1 \pmod q$ OR 
    $x^{2(m-1)} \equiv -1 \pmod p$, $x^{2(m-1)} \equiv 1 \pmod q$, as it says in Section 2.4.4 in 
    Lecture 03-05. Then $\gcd(x^{2(n-1)}-1, n)$ and $\gcd(x^{2(n-1)}+1, n)$ are the factors of $n$.
    
    
\end{enumerate}



\end{document}
