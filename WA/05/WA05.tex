
%%%%%%%%% MASTER -- compiles the summary, description, and reference sections

%%%\documentclass[11pt,onecolumn]{IEEEtran}

\documentclass[10pt]{article}


%\usepackage{course}
\usepackage{pst-all,amssymb,amsmath,amsthm,cite,url,float}
\usepackage[pdfpagelayout=OneColumn]{hyperref}

%\usepackage[usenames, dvipsnames]{color}

%\HideNotes


%\def\handout{Written Assignment 01}  % Define the title of the document. For other parameters, plz see course.sty

\newcommand{\setup}{\mathsf{Setup}}
\newcommand{\gen}{\mathsf{KeyGen}}
\newcommand{\enc}{\mathsf{Enc}}
\newcommand{\dec}{\mathsf{Dec}}
\newcommand{\param}{\mathsf{param}}
\newcommand{\pk}{\mathsf{pk}}
\newcommand{\sk}{\mathsf{sk}}
\newcommand{\adv}{\mathsf{Adv}}


% Add space between paragraphs, no indents
\setlength{\parskip}         {1ex plus 4pt minus 2pt}
\setlength{\parindent}      {0em}

% Add package for underline command \uline to work
%\usepackage{ulem}

% Set the document margins to conform to NSF standards
\setlength{\topmargin}{-.5in}
\setlength{\textheight}{9.0in}
\setlength{\textwidth}{6.5in}
\setlength{\headheight}{.25in}
\setlength{\headsep}{.25in}
\setlength{\oddsidemargin}{0in}
\setlength{\evensidemargin}{0in}

\begin{document}
\bibliographystyle{IEEEtran}

%The purpose of this document is to codify the logistics needed for students. This will be summarized in the first day of the class.




\begin{center}
	\textbf{CSE 40622 Cryptography, Spring 2018\\Written Assignment 05 (Lecture 14)}
\end{center}


Name: \textbf{Jasmine Walker}






\begin{enumerate}
	
	\item (20 pts) Formally define the second pre-image resistance and pre-image resistance of hash functions by
	designing a game and showing the relationship between the adversary's advantage and a negligible function
	of the security parameter. In other words, try to mimic what I did in Definition 2.
	
	\textbf{Answer:} \newline 
    For second pre-image resistance, the game is:
    \begin{itemize}
    \item The challenger chooses the security parameter $1^k$ and a seed $s\leftarrow KeyGen(1^k)$ for some key generation function $KeyGen$. The challenger also chooses some $x$. The challenger publishes $1^k$, $s$, and $x$.
    \item The adversary chooses a $x'$ which does not equal $x$.
    \end{itemize}
    If for all PPTA $\mathcal{A}$ there exists some negligible function $negl(k)$ such that the following is true, we say $H(.,.)$ is second pre-image resistant.
    \begin{displaymath}
    Adv^{\mathcal{A}}_{secpre} = Pr[H(x,s)=H(x',s)|x\ne x', x'\leftarrow \mathcal{A}(1^k,s)]\le negl(k)
    \end{displaymath}
    where $Adv^{\mathcal{A}}_{secpre}$ denotes the adversary's advantage.
    \newline
    \newline
    For pre-image resistance, the game is:
    \begin{itemize}
    \item The challenger chooses the security parameter $1^k$ and a seed $s\leftarrow KeyGen(1^k)$ for some key generation function $KeyGen$. The challenger also chooses some $x$ and computes $H(x,s)$. The challenger publishes $1^k$, $s$, and $H(x,s)$.
    \item The adversary chooses a $x'$.
    \end{itemize}
    If for all PPTA $\mathcal{A}$ there exists some negligible function $negl(k)$ such that the following is true, we say $H(.,.)$ is pre-image resistant.
    \begin{displaymath}
    Adv^{\mathcal{A}}_{pre} = Pr[H(x,s)=H(x',s)|x'\leftarrow \mathcal{A}(1^k,s)]\le negl(k)
    \end{displaymath}
    where $Adv^{\mathcal{A}}_{pre}$ denotes the adversary's advantage.
    
	
\end{enumerate}



\end{document}
