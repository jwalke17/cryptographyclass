
%%%%%%%%% MASTER -- compiles the summary, description, and reference sections

%%%\documentclass[11pt,onecolumn]{IEEEtran}

\documentclass[10pt]{article}


%\usepackage{course}
\usepackage{pst-all,amssymb,amsmath,amsthm,cite,url,float}
\usepackage[pdfpagelayout=OneColumn]{hyperref}

%\usepackage[usenames, dvipsnames]{color}

%\HideNotes


%\def\handout{Written Assignment 01}  % Define the title of the document. For other parameters, plz see course.sty



% Add space between paragraphs, no indents
\setlength{\parskip}         {1ex plus 4pt minus 2pt}
\setlength{\parindent}      {0em}

% Add package for underline command \uline to work
%\usepackage{ulem}

% Set the document margins to conform to NSF standards
\setlength{\topmargin}{-.5in}
\setlength{\textheight}{9.0in}
\setlength{\textwidth}{6.5in}
\setlength{\headheight}{.25in}
\setlength{\headsep}{.25in}
\setlength{\oddsidemargin}{0in}
\setlength{\evensidemargin}{0in}

\begin{document}
\bibliographystyle{IEEEtran}

%The purpose of this document is to codify the logistics needed for students. This will be summarized in the first day of the class.




\begin{center}
	\textbf{CSE 40622 Cryptography, Spring 2018\\Written Assignment 03 (Lecture 05-07)}
\end{center}


Name: \textbf{Jasmine Walker}



\begin{enumerate}
	\item (10 pts, Page 3) Prove that $x^k=x^{k\mod |\mathbb{G}|}$ for $x\in \mathbb{G}$ for any integer $k$.

	\textbf{Answer:} \newline From Theorem 1, we know that $x^{|\mathbb{G}|}=e$. $k$ can be described as some remainder $r$ plus some quotient $q$ times $|\mathbb{G}|$. So, \newline $x^{k}=x^{r+q|\mathbb{G}|}=x^{r}x^{q|\mathbb{G}|}=x^{r}e^{q}=x^{r}$ \newline We know that $k\mod |\mathbb{G}|$ will equal $r$, \newline $x^{k\mod |\mathbb{G}|}=x^{r}$ \newline $x^{k}=x^{r}=x^{k\mod |\mathbb{G}|}$ 
	
	\bigskip\item (15 pts, Page 4) In the proof of Lagrange's Theorem, I said the set $x\mathbb{H}$ cannot form a group under the same operator as in $\mathbb{G}$. Formally prove it.
	
	\textbf{Answer:} \newline I will prove that $e\not\in x\mathbb{H}$, so $x\mathbb{H}$ cannot be a group under the same operator as $\mathbb{G}$ because the identity value $e$ for the operation does not exist in $x\mathbb{H}$. \newline \textbf{Proof by contradiction:} \newline If $e\in x\mathbb{H}$, then some element $h_{1}\in\mathbb{H}$ must exist such that $x\cdot h_1=e$. By definition, $h_1$ is the inverse of $x$, and $x$ is the inverse of $h_1$. This means that, because all elements in $\mathbb{H}$ also have their inverses in $\mathbb{H}$, $x\in\mathbb{H}$. This is false because $x$ is defined $x\in \mathbb{G}- \mathbb{H}$, meaning necessarily that $x\not\in\mathbb{H}$. Then there cannot exist an $x$ such that $x\cdot h_1=e$. So, $x\mathbb{H}$ cannot exist such that $e\in x\mathbb{H}$
	
	\bigskip\item (15 pts, Page 4-5) In the proof of Lagrange's Theorem, I said $\mathbb{H}\cap x\mathbb{H}=\varnothing$. Formally prove it.
	
	
	
	\textbf{Answer:} \newline \textbf{Proof by contradiction:} \newline If $\mathbb{H}\cap x\mathbb{H}\ne\varnothing$, then there is some element that is shared by $x\mathbb{H}$ and $\mathbb{H}$. Then there must be some $h_2$ that exists in both $\mathbb{H}$ and $x\mathbb{H}$ such that, for $\exists h_1\in\mathbb{H}$, $xh_1=h_2$. Then $xh_1=h_2\Rightarrow x=h_1^{-1}h_2$. Since $h_1, h_2\in\mathbb{H}$, then $x\in\mathbb{H}$, which is false because $x$ is defined as $x\in\mathbb{G}-\mathbb{H}$. There does not exist any $h_2$ that exists in both $\mathbb{H}$ and $x\mathbb{H}$, so $\mathbb{H}$ and $x\mathbb{H}$ are disjoint. 
	
	\bigskip\item (15 pts, Page 6) Prove that any $x\in(\mathbb{G}-\{e\})$ generates $\mathbb{G}$ if $|\mathbb{G}|$ is a prime number.
	
	
	\textbf{Answer:} \newline From Theorem 1, we know that $x^{|\mathbb{G}|}=e$. From Proposition 3, we know that for some $x\in\mathbb{G}$, if $x^k=e$, then $ord(x)|k$. Since $x^{|\mathbb{G}|}=e$, we can set $k=|\mathbb{G}|$. Since $|\mathbb{G}|$ is prime, the only element that divides $k=|\mathbb{G}|$ that is also less than or equal to $k=|\mathbb{G}|$ is $|\mathbb{G}|$. So, $ord(x)$ must be $|\mathbb{G}|$ for all $x\in\mathbb{G}$. 
	
	\bigskip\item (15 pts, Page 7) Describe an algorithm for finding a generator in $\mathbb{Z}_p^*$ when $p$ is a prime number such that $p=2q+1$ for a prime $q$.
		\begin{itemize}
		\item Hint: You may use the following proposition without proving it -- An element $x\in\mathbb{G}$ is a generator of $\mathbb{G}$ if and only if $\mathsf{ord}({x})=|\mathbb{G}|$
	\end{itemize}
	
	\textbf{Answer:} \newline We know that the order of $\mathbb{Z}_p^*$ is $2q$ because the order is equal to $p-1=(2q+1)-1=2q$. So we must find an element $x\in\mathbb{Z}_p^*$ that has an order of $2q$, as stated by the proposition above. From the lecture notes and Lagrange's Theorem, we know that any element in $\mathbb{Z}_p^*$ can only produce sets with an order of $1$, $2$, $q$, and $2q$. So the algorithm to find $x$ is as follows: \newline \newline Pick a number $x$ in $\mathbb{Z}_p^*$. \newline Calculate $x^1$. \newline If $x^1\ne1$, calculate $x^2$. \newline If $x^2\ne1$, calculate $x^q$. \newline If $x^q\ne1$, then $x$ is a generator of $\mathbb{Z}_p^*$. \newline If $x$ failed any of the above tests, pick a different $x$ and try the tests again. 
	
	\bigskip\item (5pts, Page 8) Explain why $x,r$ should be non-zero in ElGamal encryption.
	
	\textbf{Answer:} \newline If $x=0$, then the attacker automatically knows because the public key $h$ will be equal to $g^0=1$. The attacker then can know the cipher $c_2$ will equal the message $m$ because $c_2=m*h^r=m*1=m$. \newline If $r=0$, the attacker will automatically knows because $c_1$ will equal $g^r=1$. Then the attacker will know the cipher $c_2$ will equal the message $m$ because $c_2=m*h^r=m*1=m$. 
	
	\bigskip\item (15 pts, Page 11) An algorithm solving DLOG problem can be used to solve CDH problem. Explain how this can be done.
	\begin{itemize}
		\item Hint: Imagine that you have an algorithm which solves the DLOG problem: It outputs $x$ given $g^x$. Even though we do not know the mechanism of that algorithm, we can still use that algorithm to as a black box (\textit{i.e.,} only see the output when we give something as input) and solve CDH problem.
	\end{itemize}
	
	
	\textbf{Answer:} \newline If the DLOG algorithm works, this is how it can be used to find the result $g^{ab}$ of the CDH problem given $g$, $g^a$, and $g^b$: \newline First, find $\mathbb{G}$ from $g$. This can be done because $g$ is a generator of $\mathbb{G}$. \newline Then, insert $\mathbb{G}$, $g$, and $g^a$ into the DLOG algorithm to get $a$. \newline Then, raise $g^b$ to $a$ to get $g^{ba}=g^{ab}$.
	
	\bigskip\item (10 pts, Page 12) Analyze why the variant of ElGamal encryption is an additive homomorphic encryption. Please explicitly show how decryption can be done after computation is conducted on the ciphertext.
	
	\textbf{Answer:} \newline The encryption of $m_1$ yields $c_{11}=g^{r_1}$ and $c_{21}=g^{m_1}g^{r_1x}$. The encryption of $m_2$ yields $c_{12}=g^{r_2}$ and $c_{22}=g^{m_2}g^{r_2x}$. We can get the encryption of $m_1 + m_2$ from the encryption of $m_1$ and the encryption of $m_2$ by computing $c_{11}\cdot c_{12}$ and $c_{21}\cdot c_{22}$. \newline $c_{11}\cdot c_{12}=g^{r_1}\cdot g^{r_2}=g^{r_1+r_2}$, and $c_{21}\cdot c_{22}=g^{m_1}g^{r_1x}\cdot g^{m_2}g^{r_2x} = g^{m_1+m_2}g^{(r_1+r_2)x}$ \newline Then the decryption can be computed by computing $(c_{11}c_{12})^x=g^{(r_1+r_2)x}$, then computing $g^{(-1)(r_1+r_2)(x)}$, then computing $(c_{21}c_{22})\cdot g^{(-1)(r_1+r_2)(x)} = g^{m_1+m_2}g^{(r_1+r_2)x}g^{(-1)(r_1+r_2)(x)} = g^{m_1+m_2}$. Then compute DLOG on $g^{m_1+m_2}$ to get $m_1+m_2$. 
\end{enumerate}



\end{document}
